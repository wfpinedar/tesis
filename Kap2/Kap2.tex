\chapter{Formulaci�n del problema}
Para el desarrollo del problema se realiza primero la definici�n del �rbol de problemas, el cual se presenta en la Figura(\ref{fig:ArProblem}), dentro del cual se expone el problema general de manera resumida, para que posteriormente sea mas sencilla la explicaci�n detallada tanto del mismo como de sus causas y sub-causas las cuales ser�n expuestas en los p�rrafos que se presentan a continuaci�n.\\

En vista de los desarrollos que se presentan hoy en d�a en el �rea de la geodesia satelital, es importante tener en cuenta que una actividad crucial para el desarrollo de los procesos que son ejecutados en esta rama de la geodesia est�n vinculados principalmente con el procesamiento posterior de los datos que son obtenidos por el sistema global de navegaci�n por sat�lite. Motivo por el cual es imprescindible que en cada proceso de an�lisis realizado en el tema de la geodesia satelital se deba contar con datos post-procesados o en su defecto que se tengan que procesar por medios propios para poder hacer un an�lisis.\\

Generalmente no se cuenta con un software que permita el procesamiento de este tipo de datos, entonces cabe la posibilidad de comprar un programa que permita este tipo de procesos a los distribuidores privados o en su defecto usar uno libre, pero siempre estar� el impedimento generado porque este programa debe ser instalado primero en el ordenador que se desea hacer el procesamiento, donde caben dos posibilidades generalmente si es libre no se puede instalar de una forma intuitiva y f�cil y de ser privativo s�lo se podr�a instalar de forma legal en un computador en el cual se tenga acceso a la licencia\cite{HOYER2008}.\\

La motivaci�n que conlleva a la elecci�n de este tema para el proyecto, es principalmente que se presenta como una necesidad latente en el �mbito de la geodesia satelital el hecho de tener a la mano de forma �gil, gratuita y accesible un medio para poder hacer el procesamiento de datos Rinex. Entonces se pretende desarrollar de una soluci�n a esta necesidad, la cual a grandes rasgos es una aplicaci�n open source de tipo web que permita acceder al procesamiento de datos de GNSS (Global Navigation Satellite System) en formato Rinex, pero sin necesidad de tener que realizar una instalaci�n en el equipo y desde cualquier lugar que se encuentra siempre y cuando tenga acceso a una conexi�n de Internet.\\

