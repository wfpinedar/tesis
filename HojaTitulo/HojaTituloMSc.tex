%\newpage
\setcounter{page}{1}
\begin{center}
\begin{figure}
\centering%
\includegraphics[height=3.5cm]{HojaTitulo/EscudoUDistrital.jpg}
%\epsfig{file=HojaTitulo/EscudoUN.eps,scale=1}%
\end{figure}
\thispagestyle{empty} \vspace*{1.5cm} \textbf{\huge
Desarrollo de una aplicaci�n web para procesamiento de datos satelitales mediante la t�cnica denominada \textquotedblleft procesamiento por punto preciso\textquotedblright.}\\[4.0cm]
\Large\textbf{Wilmar Fernando Pineda Rojas}\\[4.0cm]
\small Universidad Distrital Francisco Jos� de Caldas\\
Ingenier�a Catastral y Geodesia\\
Bogot� D.C., Colombia\\
A\~{n}o 2014\\
\end{center}

\newpage{\pagestyle{empty}\cleardoublepage}

\newpage
\begin{center}
\thispagestyle{empty} \vspace*{0cm} \textbf{\huge
Desarrollo de una aplicaci�n web para procesamiento de datos satelitales mediante la t�cnica denominada \textquotedblleft procesamiento por punto preciso\textquotedblright.}\\[2.0cm]
\Large\textbf{Wilmlar Fernando Pineda Rojas}\\[2.0cm]
\small Anteproyecto para trabajo de grado presentado como requisito parcial para optar al
t\'{\i}tulo de:\\
\textbf{Ingeniero Catastral y Geodesta}\\[2.5cm]
Director:\\
Ph.D., Ruben Javier Medina Daza\\[2.0cm]
%L\'{\i}nea de Investigaci\'{o}n:\\
%Nombrar la l\'{\i}nea de investigaci\'{o}n en la que enmarca la tesis  o trabajo de investigaci\'{o}n\\
Grupo de Investigaci\'{o}n:\\
Grupo GNU/Linux de La Universidad Distrital - Semillero En Tecnolog�a Libre\\[2.5cm]
\small Universidad Distrital Francisco Jos� de Caldas\\
Ingenier�a Catastral y Geodesia\\
Bogot� D.C., Colombia\\
A\~{n}o 2014\\
\end{center}

%SE COMENTA PARA SACAR LA DEDICATORIA Y LOS AGRADECIMIENTOS.

%\newpage{\pagestyle{empty}\cleardoublepage}

%\newpage
%\thispagestyle{empty} \textbf{}\normalsize
%\\\\\\%
%\textbf{(Dedicatoria o un lema)}\\[4.0cm]
%
%\begin{flushright}
%\begin{minipage}{8cm}
%    \noindent
%        \small
%        Su uso es opcional y cada autor podr\'{a} determinar la distribuci\'{o}n del texto en la p\'{a}gina, se sugiere esta presentaci\'{o}n. En ella el autor dedica su trabajo en forma especial a personas y/o entidades.\\[1.0cm]\\
%        Por ejemplo:\\[1.0cm]
%        A mis padres\\[1.0cm]\\
%        o\\[1.0cm]
%        La preocupaci\'{o}n por el hombre y su destino siempre debe ser el
%        inter\'{e}s primordial de todo esfuerzo t\'{e}cnico. Nunca olvides esto
%        entre tus diagramas y ecuaciones.\\\\
%        Albert Einstein\\
%\end{minipage}
%\end{flushright}
%
%\newpage{\pagestyle{empty}\cleardoublepage}
%
%\newpage
%\thispagestyle{empty} \textbf{}\normalsize
%\\\\\\%
%\textbf{\LARGE Agradecimientos}
%\addcontentsline{toc}{chapter}{\numberline{}Agradecimientos}\\\\
%Esta secci\'{o}n es opcional, en ella el autor agradece a las personas o instituciones que colaboraron en la realizaci\'{o}n de la tesis  o trabajo de investigaci\'{o}n. Si se incluye esta secci\'{o}n, deben aparecer los nombres completos, los cargos y su aporte al documento.\\
%
\newpage{\pagestyle{empty}\cleardoublepage}
%
%\newpage