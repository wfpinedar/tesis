\chapter{Objetivos}

Para este capitulo los objetivos has sido obtenidos partiendo de la definici�n del problema que se presenta en el �rbol de problemas\ref{fig:ArProblem}, en el cual se define el Problema general, luego sus principales causas y para finalizar las sub-causas. El �rbol de problemas cumple la funci�n de ser el punto de partida para la creaci�n del �rbol de objetivos \ref{fig:ArObjetivos}, que es finalmente el que proporciona tanto el objetivo general como los espec�ficos para este capitulo y que posteriormente har� posible definir algunas de las actividades que estar�n en el desarrollo de la metodolog�a del proyecto.\\

\section{Objetivo General}

\begin{itemize}
\item Dise�ar, desarrollar e implementar un software que permita un f�cil acceso al procesamiento de datos satelitales del GNSS en formato RINEX.
\end{itemize}

\section{Objetivos espec�ficos}
\begin{itemize}
\item Establecer las condiciones para una aplicaci�n que no tenga costo alguno y de licencia libre, que permita procesamiento de datos GNSS en formato RINEX.\\
\item Selecci�n de un algoritmo apropiado para procesar datos satelitales, explicaci�n te�rica del mismo y b�squeda de coherencia del algoritmo con el desarrollo del software.\\
\item Desarrollo de un software tipo web, para facilitar el acceso a los usuarios y que la capacidad de procesamiento no dependa de estos.\\
\end{itemize}

\begin{figure}[H]
\centering
\includegraphics[width=16cm]{Kap3/Fig_Kap3/ArboldeProblemas.png} %height=8cm,
\caption[�rbol de Problemas]{Gr�fica del �rbol de Problemas}
\label{fig:ArProblem}
\end{figure}

\begin{figure}[H]
\centering
\includegraphics[width=16cm]{Kap3/Fig_Kap3/ArboldeObjetivos.png} %height=8cm,
\caption[�rbol de Objetivos]{Gr�fica del �rbol de Objetivos}
\label{fig:ArObjetivos}
\end{figure}

%\section{Ejemplos de citaciones bibliogr\'{a}ficas}
%Existen algunos ejemplos para la citaci\'{o}n bibliogr\'{a}fica, por ejemplo, Microsoft Word (versiones posteriores al 2006), en el  men\'{u} de referencias, se cuenta con la opci\'{o}n de insertar citas bibliogr\'{a}ficas utilizando la norma APA (American Psychological Association) u otras normas y con la ayuda para construir autom\'{a}ticamente la lista al final del documento. De la misma manera, existen administradores bibliogr\'{a}ficos compatibles con Microsoft Word como Zotero, End Note y el Reference Manager,  disponibles a trav\'{e}s del Sistema Nacional de Bibliotecas (SINAB) de la Universidad Nacional de Colombia\footnote{Ver:www.sinab.unal.edu.co } secci\'{o}n "Recursos bibliogr\'{a}ficos" opci\'{o}n "Herramientas Bibliogr\'{a}ficas. A continuaci\'{o}n se muestra un ejemplo de una de las formas m\'{a}s usadas para las citaciones bibliogr\'{a}ficas.\\
%
%Citaci\'{o}n individual:\cite{AG01}.\\
%Citaci\'{o}n simult\'{a}nea de varios autores:
%\cite{AG12,AG52,AG70,AG08a,AG09a,AG36a,AG01i}.\\
%
%Por lo general, las referencias bibliogr\'{a}ficas correspondientes a los anteriores n\'{u}meros, se listan al final del documento en orden de aparici\'{o}n o en orden alfab\'{e}tico. Otras normas de citaci\'{o}n incluyen el apellido del autor y el a\~{n}o de la referencia, por ejemplo: 1) "...\'{e}nfasis en elementos ligados al \'{a}mbito ingenieril que se enfocan en el manejo de datos e informaci\'{o}n estructurada y que seg\'{u}n Kostoff (1997) ha atra\'{\i}do la atenci\'{o}n de investigadores dado el advenimiento de TIC...", 2) "...Dicha afirmaci\'{o}n coincide con los planteamientos de Snarch (1998), citado por Castellanos (2007), quien comenta que el manejo..." y 3) "...el futuro del sistema para argumentar los procesos de toma de decisiones y el desarrollo de ideas innovadoras (Nosella \textsl{et al}., 2008)...".\\
%
%\section{Ejemplos de presentaci\'{o}n y citaci\'{o}n de figuras}
%Las ilustraciones forman parte del contenido de los cap\'{\i}tulos. Se deben colocar en la misma p\'{a}gina en que se mencionan o en la siguiente (deben siempre mencionarse en el texto).\\
%
%Las llamadas para explicar alg\'{u}n aspecto de la informaci\'{o}n deben hacerse con nota al pie y su nota correspondiente\footnote{Las notas van como "notas al pie". Se utilizan para explicar, comentar o hacer referencia al texto de un documento, as\'{\i} como para introducir comentarios detallados y en ocasiones para citar fuentes de informaci\'{o}n (aunque para esta opci\'{o}n es mejor seguir en detalle las normas de citaci\'{o}n bibliogr\'{a}fica seleccionadas).}. La fuente documental se debe escribir al final de la ilustraci\'{o}n o figura con los elementos de la referencia (de acuerdo con las normas seleccionadas) y no como pie de p\'{a}gina. Un ejemplo para la presentaci\'{o}n y citaci\'{o}n de figuras, se presenta a continuaci\'{o}n (citaci\'{o}n directa):\\
%
%Por medio de las propiedades del fruto, seg\'{u}n el espesor del endocarpio, se hace una clasificaci\'{o}n de la palma de aceite en tres tipos: Dura, Ternera y Pisifera, que se ilustran en la Figura
%%\ref{fig:Fruto}.\\
%%\begin{figure}[h]
%%\centering%
%%\epsfig{file=Kap3/FrutoSp.eps,scale=1}%
%%\caption{Tipos y partes del fruto de palma de aceite \cite{AG03p,AG04p}.} \label{fig:Fruto}
%%\end{figure}
%
%\section{Ejemplo de presentaci\'{o}n y citaci\'{o}n de tablas y cuadros}
%Para la edici\'{o}n de tablas, cada columna debe llevar su t\'{\i}tulo; la primera palabra se debe escribir con may\'{u}scula inicial y preferiblemente sin abreviaturas. En las tablas y cuadros, los t\'{\i}tulos y datos se deben ubicar entre l\'{\i}neas horizontales y verticales cerradas (como se realiza en esta plantilla).\\
%
%La numeraci\'{o}n de las tablas se realiza de la misma manera que las figuras o ilustraciones, a lo largo de todo el texto. Deben llevar un t\'{\i}tulo breve, que concreta el contenido de la tabla; \'{e}ste se debe escribir en la parte superior de la misma. Para la presentaci\'{o}n de cuadros, se deben seguir las indicaciones dadas para las tablas.\\
%
%Un ejemplo para la presentaci\'{o}n y citaci\'{o}n de tablas (citaci\'{o}n indirecta), se presenta a continuaci\'{o}n:\\
%
%De esta participaci\'{o}n aproximadamente el 60 \% proviene de biomasa
%(Tabla \ref{EMundo1}).
%\begin{center}
%\begin{threeparttable}
%\centering%
%\caption{Participaci\'{o}n de las energ\'{\i}as renovables en el suministro
%total de energ\'{\i}a primaria \cite{AG02i}.}\label{EMundo1}
%\begin{tabular}{|l|c|c|}\hline
%&\multicolumn{2}{c|}{Participaci\'{o}n en el suministro de energ\'{\i}a primaria /\% (Mtoe)\;$\tnote{1}$}\\\cline{2-3}%
%\arr{Region}&Energ\'{\i}as renovables &Participaci\'{o}n de la biomasa\\\hline%
%Latinoam\'{e}rica&28,9 (140)&62,4 (87,4)\\\hline%
%\:Colombia&27,7 (7,6)&54,4 (4,1)\\\hline%
%Alemania&3,8 (13,2)&65,8 (8,7)\\\hline%
%Mundial&13,1 (1404,0)&79,4 (1114,8)\\\hline
%\end{tabular}
%\begin{tablenotes}
%\item[1] \footnotesize{1 kg oe=10000 kcal=41,868 MJ}
%\end{tablenotes}
%\end{threeparttable}
%\end{center}
%
%NOTA: en el caso en que el contenido de la tabla o cuadro sea muy extenso, se puede cambiar el tama\~{n}o de la letra, siempre y cuando \'{e}sta sea visible por el lector.\\
%
%\subsection{Consideraciones adicionales para el manejo de figuras y tablas}
%Cuando una tabla, cuadro o figura ocupa m\'{a}s de una p\'{a}gina, se debe repetir su identificaci\'{o}n num\'{e}rica, seguida por la palabra continuaci\'{o}n.\\
%
%Adicionalmente los encabezados de las columnas se deben repetir en todas las p\'{a}ginas despu\'{e}s de la primera.\\
%
%Los anteriores lineamientos se contemplan en la presente plantilla.\\
%
%\begin{itemize}
%\item Presentaci\'{o}n y citaci\'{o}n de ecuaciones.
%\end{itemize}
%
%La citaci\'{o}n de ecuaciones, en caso que se presenten, debe hacerse como lo sugiere esta plantilla. Todas las ecuaciones deben estar numeradas y citadas detro del texto.\\
%
%Para el manejo de cifras se debe seleccionar la norma seg\'{u}n el \'{a}rea de conocimiento de la tesis  o trabajo de investigaci\'{o}n.\\
