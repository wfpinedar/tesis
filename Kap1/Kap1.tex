\chapter*{Introducci�n}
El presente proyecto maneja una perspectiva diferente de las desarrolladas hasta la �poca en el �mbito del procesamiento de datos Rinex, es un planteamiento que se ha venido dando en los �ltimos a�os con la aparici�n del termino y las ideas de la web 2.0 tambien llamada web participativa. En esta se plantea que el conocimiento se puede poner en la web y diferentes actores pueden hacer uso de este. De esta idea de  la web deriva la importancia que tiene este proyecto, al constituirse como el primero que presenta a los usuarios de habla hispana que permite este tipo de procesamiento de datos.\\

En otros pa�ses ya se han comenzado a desarrollar varios proyectos de este tipo, algunos ya se encuentran en un estado de madures que parece imposible dado su corto tiempo de desarrollo, pero este hecho casi siempre esta dado por grandes inversiones de capital o el entusiasmo de la comunidad que lo desarrolla. Tambi�n esta el aspecto te�rico, en el cual se plantea usar una t�cnica que ofrece personalizaci�n avanzada del procesamiento de los datos, lo cual hasta el momento solo se ha usado para investigaci�n pero se pretende cambiar ese paradigma.\\

El objetivo del proyecto es proporcionar a los usuarios que no tienen la posibilidad de adquirir una aplicaci�n avanzada o comercial para procesamiento de datos satelitales, quienes no tienen el tiempo o la disposici�n para aprender una libre que puede ser algo mas compleja y finalmente a quienes no tienen estaciones de trabajo lo suficientemente poderosas para hacer este tipo de procesamiento.\\

Se espera que esta aplicaci�n llegue a todas aquellas personas que requieren usar procesamiento de datos satelitales, pero tambi�n a las industrias que quieran apoyar la iniciativa en el futuro y finalmente a las comunidades que deseen desarrollar nuevas funcionalidades y proporcionar soporte. La gran limitaci�n  que se presenta es la oposici�n que pueden llegar a generar las viejas tecnolog�as al desarrollo de esta.\\

En la parte de desarrollo de la aplicaci�n se pretende usar una metodolog�a �gil que permita avanzar de la mejor manera posible y por parte de las pruebas la metodolog�a es comparativa dado que se plantea usar unos datos de prueba y correr los mismos procesos de esta aplicaci�n en otras que existen el el mercado.